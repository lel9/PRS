\documentclass[a4paper,12pt]{report} %размер бумаги устанавливаем А4, шрифт 12пунктов
\usepackage[T2A]{fontenc}
\usepackage[utf8]{inputenc}
%\usepackage[cp1251]{inputenc}%включаем свою кодировку: koi8-r или utf8 в UNIX, cp1251 в Windows
\usepackage[english,russian]{babel}%используем русский и английский языки с переносами
\usepackage{hyperref}
%\usepackage{amssymb,amsfonts,amsmath,mathtext,cite,enumerate,float} %подключаем нужные пакеты расширений
%\usepackage[dvips]{graphicx} %хотим вставлять в диплом рисунки?
%\graphicspath{{images/}}%путь к рисункам
\setcounter{tocdepth}{4}

\usepackage{geometry} % Меняем поля страницы
\geometry{left=2cm}% левое поле
\geometry{right=1.5cm}% правое поле
\geometry{top=1cm}% верхнее поле
\geometry{bottom=2cm}% нижнее поле
\makeatletter
%\def\@makechapterhead#1{%
%	{\normalfont
		
%		\bfseries\Huge{#1}\par\nobreak\vskip 20\p@}}
%\makeatother
%\newcommand{hell}[1]{\value{chapter}\chapter{ #1 }}
\begin{document}
	\begin{titlepage}
	\newpage
	
	\begin{center}
		Министерство науки и высшего образования Российской Федерации \\
		Федеральное государственное бюджетное образовательное учреждение \\ 
		высшего образования \\
		«Московский государственный технический университет \\
		имени Н.Э. Баумана \\
		(национальный исследовательский университет)» \\
		(МГТУ им. Н.Э. Баумана)
		
		\hrulefill
	\end{center}
	
	\flushleft{ФАКУЛЬТЕТ Информатика и системы управления}
	\flushleft{КАФЕДРА Программное обеспечение ЭВМ и информационные технологии (ИУ7)}	
	\flushleft{НАПРАВЛЕНИЕ ПОДГОТОВКИ  09.04.04 Программная инженерия}
	
	\vspace{10em}
	
	\begin{center}
		\Large Методические указания \\ к лабораторным работам по курсу
	\end{center}
	
	\vspace{2.5em}
	
	\begin{center}
		\textsc{\textbf{Проектирование рекомендательных систем}}
	\end{center}
	
	\vspace{6em}
	
	\vspace{\fill}
	
	\begin{center}
		Москва 2022
	\end{center}
	
\end{titlepage}% это титульный лист
	\tableofcontents % это оглавление, которое генерируется автоматически
	
	\chapter* {Введение}
	\chapter {Лабораторная работа 1.1. Сборка и запуск учебного веб-сайта}
	\textbf{Цель}: подготовка рабочего места для лабораторных работ 1.2-1.14.
	
	
	\section{Задание}
	Скачать исходный код учебного веб-сайта из репозитория курса \url{}, изучить его, выполнить настройку окружения в соответствии с инструкцией, запустить учебный веб-сайт и проверить работоспособность.
	\section{Учебный веб-сайт}
	\subsection{Набор данных}
	В лабораторных работах будут использованы общедоступные данные веб-сайта MovieLens (\url{https://movielens.org}), собранные и опубликованные компанией GroupLens Research: \url{https://grouplens.org/datasets/movielens}. Компания предоставила несколько наборов данных, для учебного веб-сайта выбран набор ml-latest-small.zip (\url{https://files.grouplens.org/datasets/movielens/ml-latest-small.zip} от 9/2018, содержащий 100 000 оценок и 3 600 тегов, примененных к 9 000 фильмам 600 пользователями.
	
	Набор состоит из следующих файлов:
	\begin{itemize}
		\item movies.csv -- фильмы, для каждого указаны идентификатор (movieId), название и год выхода (title), список жанров (genres);
		\item ratings.csv -- оценки, для каждой указаны идентификатор пользователя (userId), идентификатор фильма (movieId), значение от 0.5 до 5 (rating), отметка времени добавления (timestamp);
		\item tags.csv -- теги, для каждого указан идентификатор пользователя (userId), идентификатор фильма (movieId), значение (tag), отметка времени добавления (timestamp);
		\item links.csv -- соответствие идентификаторов фильмов на веб-сайте MovieLens идентификаторам на других ресурсах (\url{http://www.imdb.com}, \url{https://www.themoviedb.org}), в лабораторных работах использоваться не будет.
    \end{itemize}
	\subsection{Серверная часть}
	Серверная часть написана на языке Python с использованием фреймворка Flask и состоит из следующих файлов:
	\begin{itemize}
		\item model.py -- логика запросов к базе данных (чтение и запись);
		\item model\_helpers.py -- вспомогательные функции для файла model.py, такие как установка и закрытие соединения с базой данных, преобразование объектов sqllite3.Row в словари;
		\item api.py -- API, каждая функция API вызывает соответствующую функцию из файла model.py для чтения или записи данных в базу данных.
	\end{itemize}
    В качестве СУБД используется SQLite, реализованная встроенным в Python модулем.
	\subsection{Клиентская часть}
	Про реакт (?)
	\section{Настройка окружения}
	\begin{enumerate}
		\item Установить Python 3 с официального сайта: \url{https://www.python.org/downloads/}
		\item Установить IDE (опционально), например, PyCharm: \url{https://www.jetbrains.com/pycharm/download/} 
		\item Установить ???
		\item Установить HTTP-клиент для тестирования API, например, Postman: \url{https://www.postman.com/downloads/}
		\item Перейти в директорию с исходным кодом.
		\item Установить необходимые пакеты Python (Flask), из IDE или выполнив команду
		\begin{verbatim}
			pip install -r requirements.txt
		\end{verbatim}
	    \item Запустить скрипт инициализации базы данных db\_init\_script.py, из IDE или выполнив команду
	    \begin{verbatim}
	    	python3 db_init_script.py
	    \end{verbatim}
	\end{enumerate}
    \section{Запуск и проверка работоспособности}
    \begin{enumerate}
    	\item Запустить серверную часть app.py, из IDE или выполнив команду 
    	\begin{verbatim}
    		python3 app.py
    	\end{verbatim} 
        \item Запустить клиентскую часть ???
        \item Открыть стартовую страницу в браузере, если всё настроено верно, её вид должен соответствовать рис. 1.1.
        \item В поле "идентификатор пользователя" ввести 1, в открывшейся странице должен отобразиться список фильмов (рис. 1.2)
        
    \end{enumerate}

    \chapter {Лабораторная работа 1.2. Неперсонализированные рекомендации}
    \textbf{Цель}: Изучение методов формирования неперсонализированных рекомендаций.
    
    \section{Задание}
    Доработать серверную и клиентскую часть учебного веб-сайта для отображения неперсонализированных рекомендаций: новых и самых популярных объектов.
    
    \section{Неперсонализированные рекомендации}
    Что это
    
    \section{Доработка серверной части}
    \subsection{Отображение новых фильмов}
    Для отображения новых фильмов требуется:
    \begin{enumerate}
    	\item В файле api.py создать функцию get\_new\_movies,  связанную с URI /api/movies/new, для HTTP-метода GET, которая будет вызывать функцию get\_new\_movies из модуля model.py
    	\item В файле model.py создать функцию get\_new\_movies, которая будет получать из базы данных топ-20 новых фильмов с помощью SQL-запроса
    	\begin{verbatim}
    	    SELECT * FROM movies ORDER BY year DESC LIMIT 20;
    	\end{verbatim}
    \end{enumerate}
    \subsection{Отображение самых популярных фильмов}
    Для отображения самых популярных фильмов требуется:
    \begin{enumerate}
    	\item Дополнить таблицу movies столбцом rating, для этого в файле db\_update\_script реализовать функцию add\_rating\_column, которая будет выполнять следующий SQL-запрос 
    	\begin{verbatim}
    		ALTER TABLE movies ADD rating REAL;
    	\end{verbatim}
        \item Заполнить столбец rating значениями среднего рейтинга фильмов, для этого в файле db\_update\_script реализовать функцию load\_rating, которая будет выполнять следующий SQL-запрос
        \begin{verbatim}
        	UPDATE movies SET rating = 
        	(SELECT AVG(rating) FROM ratings
        	WHERE ratings.movieId = movies.movieId);
        \end{verbatim}
        \item Модифицировать функцию row\_to\_movie в файле model\_helpers.py, добавив новое поле rating.
    	\item Модифицировать функцию get\_all\_movies в файле model.py, чтобы она возвращала фильмы, упорядоченные по рейтингу, для этого изменить SQL-запрос на следующий
    	\begin{verbatim}
    		SELECT * FROM movies ORDER BY raiting DESC;
    	\end{verbatim}
    \end{enumerate}
    
    
	
\end{document}

     
